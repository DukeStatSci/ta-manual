% Options for packages loaded elsewhere
\PassOptionsToPackage{unicode}{hyperref}
\PassOptionsToPackage{hyphens}{url}
%
\documentclass[
]{article}
\usepackage{amsmath,amssymb}
\usepackage{lmodern}
\usepackage{ifxetex,ifluatex}
\ifnum 0\ifxetex 1\fi\ifluatex 1\fi=0 % if pdftex
  \usepackage[T1]{fontenc}
  \usepackage[utf8]{inputenc}
  \usepackage{textcomp} % provide euro and other symbols
\else % if luatex or xetex
  \usepackage{unicode-math}
  \defaultfontfeatures{Scale=MatchLowercase}
  \defaultfontfeatures[\rmfamily]{Ligatures=TeX,Scale=1}
\fi
% Use upquote if available, for straight quotes in verbatim environments
\IfFileExists{upquote.sty}{\usepackage{upquote}}{}
\IfFileExists{microtype.sty}{% use microtype if available
  \usepackage[]{microtype}
  \UseMicrotypeSet[protrusion]{basicmath} % disable protrusion for tt fonts
}{}
\makeatletter
\@ifundefined{KOMAClassName}{% if non-KOMA class
  \IfFileExists{parskip.sty}{%
    \usepackage{parskip}
  }{% else
    \setlength{\parindent}{0pt}
    \setlength{\parskip}{6pt plus 2pt minus 1pt}}
}{% if KOMA class
  \KOMAoptions{parskip=half}}
\makeatother
\usepackage{xcolor}
\IfFileExists{xurl.sty}{\usepackage{xurl}}{} % add URL line breaks if available
\IfFileExists{bookmark.sty}{\usepackage{bookmark}}{\usepackage{hyperref}}
\hypersetup{
  pdftitle={Manual for Teaching Assistants: Fall 2021},
  pdfauthor={Duke University, Department of Statistical Science},
  hidelinks,
  pdfcreator={LaTeX via pandoc}}
\urlstyle{same} % disable monospaced font for URLs
\usepackage[margin=1in]{geometry}
\usepackage{longtable,booktabs,array}
\usepackage{calc} % for calculating minipage widths
% Correct order of tables after \paragraph or \subparagraph
\usepackage{etoolbox}
\makeatletter
\patchcmd\longtable{\par}{\if@noskipsec\mbox{}\fi\par}{}{}
\makeatother
% Allow footnotes in longtable head/foot
\IfFileExists{footnotehyper.sty}{\usepackage{footnotehyper}}{\usepackage{footnote}}
\makesavenoteenv{longtable}
\usepackage{graphicx}
\makeatletter
\def\maxwidth{\ifdim\Gin@nat@width>\linewidth\linewidth\else\Gin@nat@width\fi}
\def\maxheight{\ifdim\Gin@nat@height>\textheight\textheight\else\Gin@nat@height\fi}
\makeatother
% Scale images if necessary, so that they will not overflow the page
% margins by default, and it is still possible to overwrite the defaults
% using explicit options in \includegraphics[width, height, ...]{}
\setkeys{Gin}{width=\maxwidth,height=\maxheight,keepaspectratio}
% Set default figure placement to htbp
\makeatletter
\def\fps@figure{htbp}
\makeatother
\setlength{\emergencystretch}{3em} % prevent overfull lines
\providecommand{\tightlist}{%
  \setlength{\itemsep}{0pt}\setlength{\parskip}{0pt}}
\setcounter{secnumdepth}{5}
\usepackage{booktabs}
\usepackage{amsthm}
\makeatletter
\def\thm@space@setup{%
  \thm@preskip=8pt plus 2pt minus 4pt
  \thm@postskip=\thm@preskip
}
\makeatother

%% New Commands

\ifluatex
  \usepackage{selnolig}  % disable illegal ligatures
\fi
\usepackage[]{natbib}
\bibliographystyle{apalike}

\title{Manual for Teaching Assistants: Fall 2021}
\author{Duke University, Department of Statistical Science}
\date{2021-09-30}

\begin{document}
\maketitle

{
\setcounter{tocdepth}{2}
\tableofcontents
}
\hypertarget{our-purpose}{%
\section{Our Purpose}\label{our-purpose}}

As a Teaching Assistant, whatever your duties, you are part of the Statistical Science Instruction Team. This fall, we are offering nearly 1400 seats in courses to students from across the globe. Students will range from the nearly 500 undergraduates in their first 100-level course to doctoral students nearing the end of their formal coursework. We want students who take only one or two courses from the department to learn the usefulness, importance, and power of statistical thinking and modern methodologies. We aim to educate our undergraduate majors, minors and interdisciplinary majors, master's students, and Ph.D.~students in particular to be highly accomplished future leaders in statistical science, regardless of their career paths. For all of these students, whatever their programs, we aim to provide a transformative educational experience in statistical and data science. By joining our instructional team, you help us to deliver a meaningful learning experience to each of our students.

\hypertarget{the-fundamental-rule-for-tas-is}{%
\subsection{The Fundamental Rule for TAs is:}\label{the-fundamental-rule-for-tas-is}}

\textbf{Always talk with your instructor if you have questions, concerns or suggestions}. Talk with the other TAs. Talk with the DUS, the Undergraduate Coordinator, and the Departmental staff. Ask for help if you need it, ask your questions, and keep asking. Let us know what you need in order to do your job and to do it well. We're glad you are here and willing to help out. Thank you.

\hypertarget{covid-19}{%
\section{COVID-19}\label{covid-19}}

\hypertarget{what-if-you-get-sick}{%
\subsection{What if you get sick?}\label{what-if-you-get-sick}}

If you become ill or have an emergency and are unable to perform your assigned duties, please see to your own immediate needs (\url{https://studentaffairs.duke.edu/studenthealth}). Then let the instructor know as soon as you can and also let Dr.~Durso know. If you are a graduate student, please also let Lori Rauch know in the StatSci Department by emailing her at \href{mailto:lori-rauch@duke.edu}{\nolinkurl{lori-rauch@duke.edu}}. Follow the guidelines found at \url{https://coronavirus.duke.edu/}.

\hypertarget{what-if-someone-else-gets-sick}{%
\subsection{What if someone else gets sick?}\label{what-if-someone-else-gets-sick}}

What if the instructor stops showing up? Contact the DUS right away.

What if another TA stops showing up? Contact the instructor right away.

What if a student stops showing up? Tell the instructor and the DUS.

\hypertarget{responsibilties}{%
\section{Responsibilties}\label{responsibilties}}

TA responsibilities will vary between classes therefore TAs should obtain from the instructor clear guidelines for the specific tasks required of the TA. Many of these tasks are described in general terms below.

\begin{itemize}
\item
  Most TA lab duties this semester are likely to be in person while other duties may be remote. Your office hours and duties will need to accommodate changes in University policies as the semester proceeds. We are asking that all courses set up at least some Zoom office hours to accommodate students who are ill or in isolation or stuck traveling.
\item
  In general, we request that TAs working 8-15 hours per week hold two hours of office hours each week and those working less hold one hour. Your instructor may have different requirements.
\item
  We prefer that all TAs holding office hours make at least one of those hours in Zoom eah week. TAs who prefer to conduct all of their office hours remotely are welcome to do so.
\item
  Please keep informed of the latest Duke policies at keeplearning.duke.edu, keepworking.duke.edu, and coronavirus.duke.edu.
\item
  Answer all communications from the department, your instructor, other campus employers, and the university, within 24 hours, before the end of the next business day.
\item
  If you miss a scheduled meeting, training, office hours or other event, immediately contact the instructor.
\item
  Complete all requested training and learn the technology tools required for the course you are assisting.
\item
  Get in touch with the instructor before the first class and be available to meet with the instructor during the first week of class.
\item
  If you are assigned the duties of a lead TA, meet with your instructor and make sure you understand what is expected of you for that role in that course.
\item
  Attend regular meetings with the instructor (typically at least once a week).
\item
  Attend the first live or online class session (if possible and requested to do so) and discuss with the instructor whether continued attendance is required. For TAs new to a course or the TA experience, attendance throughout the semester is often expected. Ask your instructor about their policy.
\item
  Assist in assessing and reporting student performance to the instructor. This includes identifying top and/or struggling students and identifying problems or concepts causing difficulty for students. YOUR FEEDBACK IS CRUCIAL TO US!
\item
  Respond to students' e-mails, forwarding common questions to the instructor and other TAs (if any). Your instructor will set expectations about the format students will use to answer questions. If the instructor is using Piazza, ED Discussion, Sakai Conversations, or another tool for Q\&A, discuss with the instructor expectations about answers and about how to handle getting students to ask their questions in that tool rather than in emails. You may be asked to cover answering questions on particular days.
\item
  Office hours will be held in some combination of in person and online sessions for each course. We prefer that evening office hours be held online or in a more public place, such as the library, than in Old Chemistry at night. Be accessible in a reliable manner by working out the schedule of your office hours with the instructor, announcing and then holding your regular office hours. Your primary duty during the office hours is to help students from your course. See below for general guidelines for office hours.
\item
  Your other duties and responsibilities might include but are not limited to:

  \begin{itemize}
  \tightlist
  \item
    Grade labs, homework assignments, exams, etc. using Gradescope and other tools.
  \item
    If the course has a lab component, review to prepare and then run labs or assist in labs.
  \item
    Conduct review sessions prior to exams, if the instructor requests.
  \item
    Help maintain the course website or Sakai site.
  \item
    Work with student teams on their group projects.
  \item
    Improve your teaching skills by training, practice, observations and feedback.
  \end{itemize}
\item
  Keep accurate track of time spent on the course and report to the instructor if your hours are above or below the expected level.
\item
  If you are paid biweekly, update and save your timecard when you finish work for the day and submit promptly before the deadline.
\item
  Stay available for synchronous course related duties until the end of the final exam week.

  \begin{itemize}
  \tightlist
  \item
    \textbf{The academic calendar for Fall 2021 can be found here: \href{https://registrar.duke.edu/fall-2021-academic-calendar}{Fall 2021 Calendar}}
  \item
    \textbf{Fall final exams end at 10 pm on Monday, December 13th. Specific final exam schedules are available at \href{https://registrar.duke.edu/calendars-key-dates/exam-schedules}{Exam schedules for Fall 2021}.} Grades are due within 48 hours of the end of the exam period. This means TAs for are expected to continue to hold office hours during their respective final exam periods, and to be available for grading final exams in their assigned courses unless they make special arrangements with the course instructor. Please discuss the end of semester plans with the instructor you are assisting right away.
  \end{itemize}
\item
  \textbf{Always complete the course and TA evaluations in the courses for which you are enrolled as a student.}
\end{itemize}

\hypertarget{training-evaluation-and-recognition}{%
\section{Training, Evaluation, and Recognition}\label{training-evaluation-and-recognition}}

\hypertarget{training}{%
\subsection{Training}\label{training}}

As a TA, you will be trained by the department and other units of the University. All TAs will complete a mandatory training session each semester, and will get a technology check to make certain you are Zoom-ready if required. You'll get trained in other technologies as they change as well as in using Gradescope for training.

You will be assigned to STATSCI TA Fall 2021, the departmental training site for TAs. Training may be added as needed as the semester proceeds and we'll inform you there. There is required training on TA duties, on preventing harassment and on the learning technologies needed in your course.

\hypertarget{evaluation-and-recognition}{%
\subsection{Evaluation and Recognition}\label{evaluation-and-recognition}}

As a new TA or new to your course and instructor, the instructor will work with you to make sure you are ready to lead your labs and hold office hours to meet their expectations. You can expect that your instructor will virtually drop in on your labs or office hours from time to time. There will be at least one formal observation by your instructor, by the Undergraduate Coordinator, or by the Director of Undergraduate Studies during the semester. You will do training and work with them to develop your skills so that each semester that you have a TAship, you are ready for more responsibilities and leadership roles.

If your instructor does not do a midcourse survey for anonymous feedback from students and you wish to set up one for yourself, please contact Dr.~Durso for help with a Qualtrics survey.

Continuation of the opportunity to hold a Teaching Assistantship is contingent on good TA performance. Student and faculty evaluations of TAs will occur at the end of each semester via surveys. Excellent TA evaluations can also provide the basis for future recommendation letters from faculty. TAs will be able to review their evaluations at the end of each semester.

The Department of Statistical Science recognizes one doctoral student, one master's student and one undergraduate student as TAs of the Year each spring during commencement festivities. Faculty members will be asked to nominate TAs for this award each semester (for TAs in Summer 2021, Fall 2021, and Spring 2022 courses) and a departmental committee will decide on the winners based on faculty and student evaluations. TAs of the Year will be recognized at the departmental graduation at the end of the academic year. The award for the winning PhD TA is \$1,500 towards travel and computing, for the MS winning TA is \$1,000 cash award, and for the winning undergraduate is \$750 cash award.

A Qualtrics survey link has also been set up that can be shared on course sites. It does require Duke SSO authentication. Compliments from identified senders will be used for small awards, and along with Faculty feedback, for nominations for TA of the year. TA Award nomination

TAs requested a speedy feedback device. We set up an alias for this purpose: \href{mailto:stat-ta-feedback@duke.edu}{\nolinkurl{stat-ta-feedback@duke.edu}} can be given to your students for compliments, concerns and suggestions. Email comes to the DUS and the Undergrad Coordinator. TAs may also use it to let us know of issues or concerns with the TA duties, training, equipment, materials, etc. Please do not use it for mandatory reporting of harassment, etc.

\hypertarget{further-training}{%
\subsection{Further Training}\label{further-training}}

Workshops and more training opportunities are available from Learning Innovation and the Graduate School. Upcoming workshops are here \url{https://flexteaching.li.duke.edu/fall-2020/workshops/} and recorded workshops are here \url{https://flexteacdhing.li.duke.edu/fall-2020/workshops/recorded-workshops/}. (Yes, these are the correct links for Fall 2021.)

The Graduate School has many training opportunities for students interested in teaching, see their website at \url{https://gradschool.duke.edu/}. Some specific opportunities that might be of interest are

\hypertarget{certificate-in-college-teaching}{%
\subsubsection{\texorpdfstring{\href{https://gradschool.duke.edu/professional-development/programs/certificate-college-teaching}{Certificate in College Teaching}}{Certificate in College Teaching}}\label{certificate-in-college-teaching}}

The Certificate in College Teaching program both prepares you to do this and formally documents this professional development to make you more competitive when applying for positions. Students who complete the CCT will have it listed on their transcripts as an officially endorsed Duke University Graduate School certificate. The CCT combines departmental training and resources with programming from The Graduate School to give you systematic pedagogical training that not only helps you develop as a teacher, but also allows you to use your time more efficiently as you balance your research and teaching responsibilities. The Certificate in College Teaching program has three major requirements:

\begin{itemize}
\item
  Coursework (2 courses, 1 offered in the Department of Statistical Science)
\item
  Teaching experience and observation
\item
  Online teaching portfolio
\end{itemize}

The program requirements take about a year to complete, but that may vary as opportunities for gaining teaching experience vary across departments. CCT work may be done alongside other classes, research, or work on a dissertation, and should not significantly interfere with the timely completion of any of these. After you apply to the CCT program, the program director will meet with you to go over the requirements and your timeline for completing them. Discuss your CCT activities with your instructor and the Department if you need assistance scheduling observations, etc.

\hypertarget{bass-instructional-fellowships}{%
\subsubsection{\texorpdfstring{\href{https://gradschool.duke.edu/professional-development/programs/bass-instructional-fellowships}{BASS Instructional Fellowships}}{BASS Instructional Fellowships}}\label{bass-instructional-fellowships}}

The Bass Instructional Fellowship Program supports high-quality teaching experiences for Ph.D.~students where normal means of funding are unavailable. It also helps students become more knowledgeable in online college teaching. The program offers fellowships for

\begin{itemize}
\tightlist
\item
  instructors of record (Bass IORs),
\item
  instructional teaching assistants (Bass TAs), and
\item
  online apprentices (Bass OAs).
\end{itemize}

Recipients of Bass Instructional Fellowships will receive compensatory payment at the level of Arts and Sciences teaching assistants (currently \$6,000) and a scholarship covering full or partial tuition and fees for their semester of participation. This effectively ``buys out'' that much of any other existing fellowship or financial support; the student will not net any additional pay. A Bass Instructional Fellowship should not lessen a student's competitiveness for other fellowships. If the student wins another fellowship after having also won a Bass Fellowship, that other fellowship can be decreased by the amount provided by the Bass Fellowship so that there is no penalty or disadvantage to receiving a Bass at the same time (unless this arrangement is prohibited by terms of the other fellowship).

\hypertarget{teaching-ideas-series}{%
\subsubsection{\texorpdfstring{\href{https://gradschool.duke.edu/professional-development/programs/teaching-ideas-series}{Teaching IDEAS Series}}{Teaching IDEAS Series}}\label{teaching-ideas-series}}

Instructional Development for Excellence And Success is an annual workshop series open to Duke graduate students, postdocs, faculty, and staff. Invited speakers in this series will draw upon their experience to address topics relevant to classroom teaching, dealing with students, or faculty life and career paths. You will learn how to:
- recognize the complex dimensions of classroom teaching and faculty life,
- analyze difficult teaching situations and respond creatively,
- improve your teaching skills by drawing upon various resources at Duke,
- plan and design courses with the student perspective in mind, and
- engage in scholarly conversations about college teaching and learning.

\hypertarget{duke-learning-innovation}{%
\subsubsection{\texorpdfstring{\href{https://learninginnovation.duke.edu}{Duke Learning Innovation}}{Duke Learning Innovation}}\label{duke-learning-innovation}}

Learning Innovation is not part of the Graduate School but is dedicated to improving learning at Duke. \url{https://learninginnovation.duke.edu/about/} Learning Innovation provides many opportunities for professional development, including training on Sakai, Gradescope, Piazza, and other instructional tools. They also provide drop-in office hours for help with Sakai or Gradescope as well as workshops and seminars to assist you improve your careers.

\hypertarget{guidelines}{%
\section{Guidelines}\label{guidelines}}

\hypertarget{before-classes-begin}{%
\subsection{Before Classes Begin}\label{before-classes-begin}}

\begin{itemize}
\item
  Trinity resources: The undergraduate bulletin (\url{https://registrar.duke.edu/university-bulletins/undergraduate-instruction}) and T-Reqs (Trinity Requirements) (\url{https://trinity.duke.edu/undergraduate/academic-policies} are sources for Trinity College Academic Policies and Procedures. You are encouraged to review this site for information on university policy on religious holidays, disability accommodations, class attendance, credits, exams and grading, harassment, plagiarism, etc. For graduate courses, use the Graduate Bulletin \url{https://registrar.duke.edu/university-bulletins/graduate-school} and the Graduate School's policy site \url{https://gradschool.duke.edu/academics/academic-policies}.
\item
  Technology: During training, you will be asked about your ability to use Gradescope for grading and Zoom and whiteboarding for online meetings. It's fine to use a do-it-yourself hack to turn your cell phone into a document camera aimed at paper or a whiteboard. If needed for a particular course or online office hours, let us know if you lack a tablet and stylus or something to use for written explanations; it is possible to acquire an inexpensive drawing device if you need one for whiteboarding. Also let us know your Wi-Fi situation. Talk with your instructor and Dr.~Durso.
\item
  Books: Many courses now have electronic versions of the text available for free online or through the Duke Libraries. Please check with your instructor first. For some courses, the Assistant to the DUS Mrs.~Karen Whitesell (\href{mailto:karen.whitesell@duke.edu}{\nolinkurl{karen.whitesell@duke.edu}}) has physical desk copies of most books that you may use for TAing the course, but you must return the book at the end of the semester. Please contact her if you need a textbook copy for the course.
\item
  Course website: Familiarize yourself with the website for the course. This may be a site designed by the instructor or it may be a Sakai site. Sakai is a university wide course development and presentation platform; all courses use Sakai at least for grades. You will need a NetID to access Sakai at \url{http://sakai.duke.edu}. It allows faculty (and TAs if the faculty gives permission) to post course information such as lecture notes, assignments, announcements, exam solutions, and other teaching materials to which you want your students to have access. It also will also allow you to send emails to individuals or groups, host online discussions, collect homework assignments, and allow students to view their grades. For an overview of capabilities and various tutorials go to \url{http://support.sakai.duke.edu/sakai-basics}. Your TA training may include Sakai training as well. If you have problems with Sakai, let your instructor know. Note that undergraduate TAs are not supposed to have access to the gradebook in Sakai; however, you may be able to do anonymous grading. Sakai is also able to organize Zoom meetings for the course lectures, labs, and office hours. Check with your instructor or Dr.~Durso. Learning Innovation also offers training on Sakai.
\item
  Discussion tools: Piazza is no longer supported by Duke University but individual instructors may still choose to use it. Alternative discussion tools are ED Discussion, Conversations in Sakai, and traditional discussion Forums in Sakai. Once your instructor has chosen which tool they will use, we will make training available on that tool.
\item
  Gradescope: Gradescope is an online grading tool that is usually integrated with
  Sakai. Gradescope is used extensively in the department and all TAs will be trained on its use. If your instructor is using it, they will explain what they expect. If you have never used it, be sure to ask any questions you have so that you fully understand how to grade in Gradescope. You will access Gradescope using your NetID similarly to accessing Sakai. If you have previously used Gradescope for grading and think it would assist in your course duties this semester, discuss this with your instructor. If you or the instructor would like help with Gradescope, ask Joan Combs Durso, the Coordinator of Undergraduate Training, Research, and Development. We have a training site set up and TAs will provide sample documents for grading training. Over the course of the semester, we will put in place more procedures for anonymizing grading.
\item
  Copies: Copy procedures are changing. If your instructor asks you to make physical copies for the course, please contact Mrs.~Whitesell and cc: Dr.~Durso on the email. We'll get you access and instructions.
\item
  Supplies: Please let your instructor know if you have any unmet supply needs such as chalk, dry-erase markers, red pens, etc. These are only for instructional and grading purposes, not for your personal use. If these items are missing in a TA office hours space, please let Mrs.~Whitesell and Dr.~Durso know.
\item
  Computing labs: Most introductory undergraduate labs in StatSci are held virtually as the department has shifted from physical computing labs to bring-your-own-device labs. In remote teaching, the situation is complicated by distance and the need to see the screen of your students. Graduate Teaching Assistants (and some experienced undergraduates) usually supervise the computing lab sections; undergraduate TAs may be lab helpers, so it is important that you check out the virtual lab or classroom environment well in advance of the class to make sure you understand how it will function for students. Get help with the Zoom environment from Dr.~Durso or from Learning Innovation's online training options.
\item
  Collaboration Tools: Whether your instructor is using email, Slack, Basecamp, Microsoft
  Teams, or any other collaboration tool, make sure you understand how to use it and what your instructor's preferred communication methods are.
\end{itemize}

\hypertarget{professionalism}{%
\subsection{Professionalism}\label{professionalism}}

\begin{itemize}
\item
  Whether you are attending a live class session with the instructor, leading a lab, or holding office hours online, please recall that you are a leader of students and representing the department. Students may show up to online sessions in pajamas, wrapped in blankets, and attending their session from their beds, but the TA shouldn't. If students show up inappropriately dressed, please discuss the issue with the instructor.
\item
  Use an appropriate virtual background if needed and if your computer can handle it. Try to be in a boring environment, not in a distracting one. A bookcase or office space, a living room or even a yard are fine backgrounds. Be sure to have adequate lighting. Sometimes it is just a matter of turning to face a different direction. No need to buy a ring light, but there are many YouTube videos that show you how to set up a good environment for leading an online session.
\item
  For online sessions, if your environment is noisy, use a headset and mute yourself when you are not speaking. Practice in advance where you know you will be, so you can have someone listen to you and make sure the sound is adequate.
\item
  If you need help with this, please contact Dr.~Durso.
\end{itemize}

\hypertarget{labs}{%
\subsection{Labs}\label{labs}}

\begin{itemize}
\item
  Meet with the course instructor before courses begin and at least once a week during the semester to learn how the instructor would like the lab session run and to discuss upcoming labs and any issues that arise during the semester. First year graduate students have usually not learned about statistics from the same perspective as our undergrad courses, and some have never used the software that they will be teaching. Don't hesitate to ask questions. Undergraduates may have taken the course from a different instructor whose approach is different. Talk to the instructor of this course!
\item
  During the first day of lab, do what you can to set the climate of the course and develop a rapport with the class. Get students to interact with you and their classmates either through introductions, small group discussion, or asking questions. Give information about yourself that helps establish that you are both credible and approachable. You will want students to leave after the first class knowing why the lab sections are important and what your expectations will be. You will want to leave after the first class knowing the students' expectations and knowing that they are engaged enough to begin expending the time and energy needed to do well in the course.
\item
  Make sure you have worked through the lab ahead of time leaving plenty of time to ask the instructor for clarification. Make sure you understand the general learning objectives. Be ready to explain to students why they are being asked to do the exercise.
\item
  During lab, engage the students. Encourage them to explore and play rather than simply go through the motions. Ask them questions. Be receptive to their questions. Help them make the link between the computing lab and what has been covered in the text and lectures.
\item
  Check in with breakout groups online, even when it looks like there are no questions. Wander around the live classroom or lab. Students are more likely to reach out and ask a question if you're physically (or virtually) closer to them than if you're sitting in front of the classroom. Don't ask the Ferris Bueller query ``Any questions?'' Instead, ``what questions do you have for me?'' or ``I'll take 3 questions now'' are both better ways to solicit inquiry.
\item
  Please avoid \emph{``didactic dictation''}. One of the most frustrating experiences for a new user of R is to have code quickly and sometimes inaudibly dictated to them by an experienced coder who talks too fast while typing in unreadably tiny font on a screen half the class can't see, and allows no time for correction, troubleshooting, or questions. (Use .Rmd files for that instead.) Live coding, where you are demonstrating something as you talk about it, and students code along, should be done slowly. You don't have to be perfect. It's okay to make mistakes and correct them, modeling authentic coding for your students. Demonstrate first, and then have students work on their own code. Watch for frustration. Address it positively.
\end{itemize}

Excellent suggestions for how to do this well can be found \href{https://mine-cetinkaya-rundel.github.io/golive-uscots/golive.html\#1}{here}.

\begin{itemize}
\item
  If you have suggestions for improving the lab materials or find errors in the lab instructions, please discuss them with your instructor.
\item
  It may be a hardship for some students to have to use their cameras during an online lab session, but make sure to set expectations. You can't see how they are doing if their cameras off the whole session. Try at least starting with cameras on for a check-in. Talk with the instructor about setting expectations. Do activities to keep them engaged even if the cameras are not able to be used.
\end{itemize}

\hypertarget{office-hours}{%
\subsection{Office Hours}\label{office-hours}}

\begin{itemize}
\item
  Many instructors will set up Zoom office hours room in Sakai. Practices may differ between instructors but it is a good idea to set up a Google sheet that stays linked in Sakai that can be used for students to indicate their question or need. This will help you manage the crowd that could show up. You will find an example of this in the TA training site in Sakai.
\item
  Make sure you have read the text and class notes. Use the same notation, definitions, and perspectives as the author of the text and the instructor of the class.
\item
  Make sure to convey your interest in helping students. Sitting in the help room (or on Zoom) doing your own homework and not making eye contact with an entering student will often convey to the student that you are not interested in helping.
\item
  Avoid solving homework problems for students. Giving students the answer does not help them develop the problem-solving skills that will be necessary for them to do well in the course. Ask guiding, probing questions, but get the student to do the thinking. Offer similar problems as examples to work through, together.
\item
  Students differ in their mathematical preparation, and for our undergraduate courses, students will often be very rusty on algebra or calculus, depending on the course. Be sure to work from the level of the student. If a student is in trouble beyond your abilities, refer them to the instructor.
\item
  Make sure to speak clearly, slowly, and audibly, whether in person or online. Do not assume that your students knows a particular theorem or distribution if it has not yet been used in the course. Speak respectfully and let them know the course is meant to be challenging but success happens with practice and spending sufficient time on the work.
\item
  When you don't know how to work to the answer, do NOT fake it. Be honest, and seek another TA or the instructor. Giving a student incorrect information will snowball into a larger problem. If you promise to get back to them, write it down and follow it up. Share with your peer TAs or ask in a discussion forum in the TA training site.
\item
  Let the instructor know if you have students who are struggling with the material or who don't know where to begin. It's hard for anyone to ask for help if they are not used to needing it. Be kind and patient with students who show up feeling panicked or anxious. Reach out to Duke Reach if you suspect they need help, and let the instructor know. But mostly, be kind. You were a beginner once, and the student genuinely may be lost or have not learned how to study in an online course. Help them, and get help with this if it's a problem. Kindness and patience are the best compliments in TA evaluations.
\end{itemize}

\hypertarget{grading}{%
\subsection{Grading}\label{grading}}

\begin{itemize}
\item
  Set grading policies for partial credit, late papers, etc. with the instructor. Make sure these policies are easily accessible by students (e.g., on Sakai). Make sure you understand the instructor's expectations.
\item
  Find out from the instructor whether they expect you to make the solution key or grading rubric, including both answers and a breakdown of how points will be distributed. Get the instructor or a more senior TA to review before it goes live.
\item
  Before you begin to grade, go through solutions and double check answers, points, etc.
\item
  Grade by question, i.e., grade all of question 1, then all of question 2,\ldots{} This will facilitate greater consistency. You may be assigned only one or a few questions from an entire assignment, especially if you are using Gradescope.
\item
  If you are sharing grading duties, make sure to calibrate for consistency. Students will be frustrated and angry if graders differ in their process of assigning points (and you will hear about it!).
\item
  We use \emph{positive grading} in the department. For each problem, clearly mark the number of points earned and the number of points possible. Gradescope or Sakai will do the tallying for you.
\item
  Record grades as per instructions of the instructor, but always double check your work.
\item
  Check with the instructor on how to enter missing grades, i.e.~with a blank or a zero. Blanks can play havoc with scoring algorithms that require replacing minimums. This also depends on whether you are grading with Gradescope or Sakai.
\item
  Grading turnaround deadlines will be set by the instructor. Be sure you understand their expectations and that you communicate your time availability. Gradescope shows what you have already graded and haven't.
\end{itemize}

\hypertarget{tas-and-students}{%
\section{TAs and Students}\label{tas-and-students}}

\hypertarget{academic-misconduct}{%
\subsection{Academic Misconduct}\label{academic-misconduct}}

It's crucial to let the course instructor know if you suspect or receive reports of academic misconduct so that we may handle all cases of dishonesty according to University Policy. It is your duty to report your suspicions to the instructor and they will investigate. Do not interrogate the students. This typically requires that the instructor report offenses to the Office of Judicial Affairs who will advise on mediation, arbitration, informal resolution or disciplinary action. The Duke Standard is as follows:

\begin{quote}
Duke University is a community dedicated to scholarship, leadership, and service and to the principles of honesty, fairness, respect, and accountability. Citizens of this community commit to reflect upon and uphold these principles in all academic and non-academic endeavors, and to protect and promote a culture of integrity.

To uphold the Duke Community Standard:

\begin{itemize}
\item
  I will not lie, cheat, or steal in my academic endeavors;
\item
  I will conduct myself honorably in all my endeavors; and
\item
  I will act if the Standard is compromised.
\end{itemize}
\end{quote}

Faculty and TAs are encouraged to be proactive in preventing dishonesty. For online exams, randomization and unique problems can help. Some time-limiting techniques are helpful but students with accommodations must be helped as well. There are hundreds of companies out there willing to sell students answers to problems sets, quizzes, and exams; these are all a violation of the Duke Community Standard. If you suspect such usage, talk with the instructor

\hypertarget{learning-resources-for-students}{%
\subsubsection{Learning Resources for Students}\label{learning-resources-for-students}}

The following resources are available through StatSci and the university for students having difficulty in a course.

\begin{itemize}
\item
  Direct interaction with course instructor(s), Teaching Assistant(s)
\item
  Scheduled Office hours of instructor(s) and course-specific TA(s)
\item
  Academic Resource Center, \url{http://web.duke.edu/arc}, including peer tutors and group study sessions for some introductory courses.
\item
  Let the instructor and then the Director of Undergraduate Studies and the Undergraduate Coordinator know if you need help with a particular student.
\end{itemize}

\hypertarget{group-work}{%
\subsection{Group Work}\label{group-work}}

If your course allows group work for homework or labs, ask the instructor to speak with your students about what type of collaboration is appropriate. For instance, if students can work together on HW problems, discussion is appropriate, but students must write up individual responses for every question. Student answers should not be the same as their collaborators'. If there is an in-class group assignment for which one submission is expected, make certain the submission instructions are clear and followed by the groups. Our students come from all over the world, from all kinds of prior learning experiences, and what is permitted in group work can be highly varied. Please help your students understand the nuances of permissible group work.

\hypertarget{students-with-disabilities}{%
\subsection{Students with Disabilities}\label{students-with-disabilities}}

Undergraduate students (Trinity College and the Pratt School of Engineering) must request a formal \emph{Professor Accommodation Letter} each semester from the Student Disability Access Office that lists the student's approved accommodation(s). This is needed in summer, as well. The student is responsible for emailing the instructor, forwarding the official letter, and requesting a meeting to discuss how the approved accommodations will be implemented in the class. In some classes, one TA will be asked to manage accommodations for the course. In larger classes, multiple TAs may be involved. If your instructor asks for assistance in providing accommodations for a student, be sure that all information regarding a student's disability or accommodations is kept confidential. Do not discuss it with other TAs or in front of other students. Questions should be directed to the instructor or to the Student Disability Access Office at (919) 668-1267. If a student approaches you about accommodations, ask them to communicate with the instructor and let the instructor know you have referred them. If you are leading a lab and a student asks for lab accommodations, discuss it with the instructor.

The Academic Resource Center's Testing Center is able to help with accommodations for remote testing, as well as in person. If the instructor decides to use it, students with accommodations must request testing arrangements at least one week in advance. Here is the website for information \url{https://arc.duke.edu/about-arc/testing-center}

\hypertarget{diversity-and-inclusiveness}{%
\subsection{Diversity and Inclusiveness}\label{diversity-and-inclusiveness}}

Students from all diverse backgrounds and perspectives need to be well-served by our courses, their learning needs to be addressed both in and out of class, and the diversity that the students bring to this class need to be viewed as a resource, strength, and benefit. Therefore, the materials and activities you present need to be respectful of diversity: gender identity, sexuality, disability, age, socioeconomic status, ethnicity, race, nationality, religion, and culture. Our department's statement is here: \url{https://stat.duke.edu/diversity-equity-and-inclusion}.

If you have suggestions for improving the effectiveness of the course for you personally, or for other students or student groups, let the faculty member or the DUS know. There is a module on this in TA training in Sakai.

Furthermore, you should strive to create a learning environment for our students that supports a diversity of thoughts, perspectives and experiences, and honors students' identities (including gender identity, sexuality, disability, age, socioeconomic status, ethnicity, race, nationality, religion, and culture.) To help accomplish this, encourage your students to

\begin{itemize}
\item
  let you know if they have a name and/or set of pronouns that differ from those that appear in the Duke records,
\item
  reach out if you feel like their performance in the class is being impacted by their experiences outside of class, and remind them that if they prefer to speak with someone outside of the course, their academic dean/director is an excellent resource, and
\item
  talk to you if something was said in class (by anyone) that made them feel uncomfortable.
\end{itemize}

Requiring people to identify their pronouns is not helpful; invite it but don't force it. If you want to give your pronouns, do so. If offered, thank the student, and try to respect that request. The DukeHub now lets students who wish to identify their personal gender pronouns to do so in their records: \url{https://studentaffairs.duke.edu/csgd/pronouns/dukehub-20-pronouns-tutorial}

Please let the instructor know if there is an issue that you notice in class or in office hours that is disrespectful.

\hypertarget{harassment}{%
\subsection{Harassment}\label{harassment}}

Harassment complaints that involve both undergraduate students and either graduate students, faculty members or employees are addressed, as necessary, through the intervention of Office of Institutional Equity or the official responsible for the respondent's supervision. Such situations are of particular concern because the possible inherent power differential between the parties increases the potential for coerciveness. See \url{http://www.duke.edu/web/equity/harassment.html} for more information on Duke's harassment policy. If you have any concerns regarding harassment, please feel free to reach out to the instructor or DUS for advice as well.

\hypertarget{relationships-with-students}{%
\subsection{Relationships with Students}\label{relationships-with-students}}

Consensual romantic or sexual relationships between any student charged with academic instruction and students receiving such instruction are prohibited. This applies to teaching assistants, research assistants, tutors, graders and any other students who provide academic instruction to any other student. A TA could lose the opportunity to continue in the instruction role and such behavior could also violate student conduct policies. This is covered in more detail in Appendix Z to the Faculty Handbook here: \url{https://provost.duke.edu/sites/all/files/FHB_App_Z.pdf}

If you are already in a relationship with a student who is in the class for which you are TAing, please discuss this with the instructor and the DUS or Undergraduate Coordinator. We can change your assignment or make certain you are never grading that person's work or helping in their lab.

\hypertarget{emergency-conditions}{%
\subsection{Emergency Conditions}\label{emergency-conditions}}

Download and install the Live Safe app to get emergency notifications for Duke's campuses. You can obtain information on emergency notifications and procedures at \url{https://emergency.duke.edu}. Please become familiar with this site and its contents. Instructors may be covering these procedures in class.

Our academic calendar sometimes brings with it the threat of severe weather in North Carolina. At the start of your course, be sure and locate the nearest tornado safe zone in your assigned location or wherever you are staying. This also applies if you are teaching your lab from your own residence. If you are in Old Chem, the nearest tornado safe area is in the library buildings next door. In other buildings, if you can't find signage, ask staff in that building for their tornado safe location. If there is a tornado watch, simply monitor conditions. If there is a tornado warning announced, however, you will be notified to take shelter in the nearest tornado safe area.

\hypertarget{students-in-distress}{%
\subsection{Students in Distress}\label{students-in-distress}}

Our students can be under a lot of stress. Any lab absences, missed deadlines, or concerning behavior should be brought to your instructor's attention. In addition, familiarize yourself with the help offered by Duke Reach \url{https://studentaffairs.duke.edu/dukereach1} as well as the Duke Counseling and Psychological Services \url{https://studentaffairs.duke.edu/caps}. You should also read Recognizing and Responding to Students in Distress, from the Office of the Provost.

First year students and visiting students, in particular, may not be aware of these resources. Let the instructor know if you are concerned about a student's behavior or attendance or sudden changes in quality and timeliness of assignments. You can also let the Director of Undergraduate Studies (Professor Çetinkaya-Rundel) or the Undergraduate Coordinator (Joan Combs Durso) of your concerns. If you are holding office hours on nights or weekends, virtual campus resources may be more lightly staffed, so you should also know about other options for help. Check with Karen Whitesell in the Department of Statistical Sciences or contact the Dean of the Day, or in the event of an emergency requiring transportation, call campus police, or 911.

If you find yourself in distress, talk with the instructor, Professor Çetinkaya-Rundel, or Dr.~Durso. Let us know immediately if you need help. Please.

  \bibliography{book.bib,packages.bib}

\end{document}
